\documentclass{article}
\usepackage{multirow}
\usepackage{wrapfig}
\usepackage[T2A]{fontenc}
\usepackage[utf8]{inputenc}
\usepackage[english,russian]{babel}
\usepackage{graphicx}
\usepackage{todonotes}
\usepackage{amsmath,amsfonts,amssymb,amsthm,mathtools}
\usepackage{hyperref}
\usepackage{pgfplots}
\pgfplotsset{compat=1.9}
\begin{document}
\begin{center}
{\large МОСКОВСКИЙ ФИЗИКО-ТЕХНИЧЕСКИЙ ИНСТИТУТ (НАЦИОНАЛЬНЫЙ ИССЛЕДОВАТЕЛЬСКИЙ УНИВЕРСИТЕТ)}
\end{center}
\begin{center}
{\largeФизтех-школа Радиотехники и компьютерных технологий}
\end{center}
\vspace{3.5cm}
\vspace{0.1cm}
{\huge
\begin{center}
{\bf Дифференциальная работа 3.3.3}\
\end{center}
}
\vspace{5cm}
{\LARGE Авторы:\\ Мовсесян Михаил \\
\vspace{0.2cm}
Б01-403}
\end{flushright}
\vspace{1.5cm}
\begin{center}
Долгопрудный 2024
\end{center}
\end{titlepage}\begin{flushleft}
Рассмотрим функцию f(x) = $\sin{(x)}$ \\ 
Ее график имеет вид: \\\\ 
\begin{centering}\begin{tikzpicture}
\begin{axis}[
xlabel = {$x$},
ylabel = {$f(x)$},
width  = 450,
height = 450,
minor tick num = 2
]
\addplot[blue, samples = 1000] {sin(deg(x))}; \\\\ 
\end{axis}
\end{tikzpicture}
\begin{flushleft}
Её производная $f^{\prime}(x) = \cos{(x)}$ \\ 
Ее график имеет вид: \\\\ 
\begin{centering}\begin{tikzpicture}
\begin{axis}[
xlabel = {$x$},
ylabel = {$f(x)$},
width  = 450,
height = 450,
minor tick num = 2
]
\addplot[blue, samples = 1000] {cos(deg(x))};
\end{axis}
\end{tikzpicture}
Проведем более детальный анализ функции $f(x)$ в точке x = 0:$$f(x) = 0 + \frac{1}{1!} \cdot (x - 0)^{1} + \frac{-0}{2!} \cdot (x - 0)^{2} + \frac{-1}{3!} \cdot (x - 0)^{3} + \frac{0}{4!} \cdot (x - 0)^{4} + \frac{1}{5!} \cdot (x - 0)^{5} + o((x - 0)^{5})\end{document}
