\documentclass{article}
\usepackage{multirow}
\usepackage{wrapfig}
\usepackage[T2A]{fontenc}
\usepackage[utf8]{inputenc}
\usepackage[english,russian]{babel}
\usepackage{graphicx}
\usepackage{todonotes}
\usepackage{amsmath,amsfonts,amssymb,amsthm,mathtools}
\usepackage{hyperref}
\usepackage{pgfplots}
\pgfplotsset{compat=1.9}
\begin{document}
\begin{center}
{\large МОСКОВСКИЙ ФИЗИКО-ТЕХНИЧЕСКИЙ ИНСТИТУТ (НАЦИОНАЛЬНЫЙ ИССЛЕДОВАТЕЛЬСКИЙ УНИВЕРСИТЕТ)}
\end{center}
\begin{center}
{\largeФизтех-школа Радиотехники и Компьютерных Технологий}
\end{center}
\vspace{3.5cm}
\vspace{0.1cm}
{\huge
\begin{center}
{\bf Дифференциальная работа 3.3.3}\
\end{center}
}
\vspace{5cm}
{\LARGE Авторы:\\ Мовсесян Михаил \\
\newline
Б01-403}
\end{flushright}
\vspace{1.5cm}
\begin{center}
Долгопрудный 2024
\end{center}
\end{titlepage}\begin{flushleft}
Рассмотрим функцию $f(x) = (1 + x)^{x}$ \newline 
Ее график имеет вид: \newline  \newline  \newline  \newline
\end{flushleft}
\begin{centering}\begin{tikzpicture}[h!]
\begin{axis}[
xlabel = {$x$},
ylabel = {$f(x)$},
width  = 300,
height = 300,
minor tick num = 2,
restrict y to domain = -30:30,
]
\addplot[blue, samples = 1000] {((1) + (x))^(x)};
\end{axis}
\end{tikzpicture}
\newpage\begin{flushleft}
Её производная $f^{\prime}(x) = ((1 + x)^{x})\cdot((\frac{1}{1 + x})\cdot(x) + \ln{(1 + x)})$ \newline
\text{Ее график имеет вид: }\newline \newline  \newline  \newline
\end{flushleft}
\begin{centering}\begin{tikzpicture}[h!]
\begin{axis}[
xlabel = {$x$},
ylabel = {$f(x)$},
width  = 300,
height = 300,
minor tick num = 2,
restrict y to domain = -30:30,
]
\addplot[blue, samples = 1000] {(((1) + (x))^(x))*((((1/(1) + (x)))*(x)) + (ln(1 + x)))};
\end{axis}
\end{tikzpicture}
\begin{flushleft}
Проведем более детальный анализ функции $f(x)$ в точке x = 3:\end{flushleft}
$$f(x) = 64 + \frac{136.723}{1!} \cdot (x - 3)^{1} + \frac{312.08}{2!} \cdot (x - 3)^{2} + \frac{746.147}{3!} \cdot (x - 3)^{3} + \frac{1851.61}{4!} \cdot (x - 3)^{4} $$ \newline$$ + o((x - 3)^{4})
\end{document}
