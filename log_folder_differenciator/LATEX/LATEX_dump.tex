\documentclass{article}
\usepackage{multirow}
\usepackage{wrapfig}
\usepackage[T2A]{fontenc}
\usepackage[utf8]{inputenc}
\usepackage[english,russian]{babel}
\usepackage{graphicx}
\usepackage{todonotes}
\usepackage{amsmath,amsfonts,amssymb,amsthm,mathtools}
\usepackage{hyperref}
\usepackage{pgfplots}
\pgfplotsset{compat=1.9}
\begin{document}
\begin{center}
{\large МОСКОВСКИЙ ФИЗИКО-ТЕХНИЧЕСКИЙ ИНСТИТУТ (НАЦИОНАЛЬНЫЙ ИССЛЕДОВАТЕЛЬСКИЙ УНИВЕРСИТЕТ)}
\end{center}
\begin{center}
{\largeФизтех-школа Радиотехники и Компьютерных Технологий}
\end{center}
\vspace{3.5cm}
\vspace{0.1cm}
{\huge
\begin{center}
{\bf Дифференциальная работа 3.3.3}\
\end{center}
}
\vspace{5cm}
{\LARGE Авторы:\\ Мовсесян Михаил \\
\newline
Б01-403}
\end{flushright}
\vspace{1.5cm}
\begin{center}
Долгопрудный 2024
\end{center}
\end{titlepage}\begin{flushleft}
Рассмотрим функцию $f(x) = (x + 1)^{\frac{3}{x}}$ \newline 
Ее график имеет вид: \newline  \newline  \newline  \newline
\end{flushleft}
\begin{centering}\begin{tikzpicture}[h!]
\begin{axis}[
xlabel = {$x$},
ylabel = {$f(x)$},
width  = 300,
height = 300,
minor tick num = 2,
restrict y to domain = -30:30,
]
\addplot[blue, samples = 1000] {((x) + (1))^((3/x))};
\end{axis}
\end{tikzpicture}
\newpage\begin{flushleft}
Её производная $f^{\prime}(x) = ((x + 1)^{\frac{3}{x}})\cdot((\frac{1}{x + 1})\cdot(\frac{3}{x}) + (\ln{(x + 1)})\cdot(\frac{-3}{(x)\cdot(x)}))$ \newline
\text{Ее график имеет вид: }\newline \newline  \newline  \newline
\end{flushleft}
\begin{centering}\begin{tikzpicture}[h!]
\begin{axis}[
xlabel = {$x$},
ylabel = {$f(x)$},
width  = 300,
height = 300,
minor tick num = 2,
restrict y to domain = -30:30,
]
\addplot[blue, samples = 1000] {(((x) + (1))^((3/x)))*((((1/(x) + (1)))*((3/x))) + ((ln(x + 1))*((-3/(x)*(x)))))};
\end{axis}
\end{tikzpicture}
\begin{flushleft}
Проведем более детальный анализ функции $f(x)$ в точке x = 1:\end{flushleft}
$$f(x) = 8 + \frac{-4.63553}{1!} \cdot (x - 1)^{1} + \frac{5.95708}{2!} \cdot (x - 1)^{2} + \frac{-11.0558}{3!} \cdot (x - 1)^{3} + \frac{26.5948}{4!} \cdot (x - 1)^{4} $$ \newline$$ + \frac{-78.2852}{5!} \cdot (x - 1)^{5} + o((x - 1)^{5})
\end{document}
